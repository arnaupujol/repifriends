\nonstopmode{}
\documentclass[a4paper]{book}
\usepackage[times,inconsolata,hyper]{Rd}
\usepackage{makeidx}
\usepackage[utf8]{inputenc} % @SET ENCODING@
% \usepackage{graphicx} % @USE GRAPHICX@
\makeindex{}
\begin{document}
\chapter*{}
\begin{center}
{\textbf{\huge Package `epifriends'}}
\par\bigskip{\large \today}
\end{center}
\inputencoding{utf8}
\ifthenelse{\boolean{Rd@use@hyper}}{\hypersetup{pdftitle = {epifriends: Epidemiological Foci Relating Infections by Distance (EpiFRIenDs)}}}{}\begin{description}
\raggedright{}
\item[Type]\AsIs{Package}
\item[Title]\AsIs{Epidemiological Foci Relating Infections by Distance
(EpiFRIenDs)}
\item[Version]\AsIs{0.1.0}
\item[Author]\AsIs{Mikel Majewski, Arnau Pujol}
\item[Maintainer]\AsIs{Arnau Pujol }\email{arnau.pujol@isglobal.org}\AsIs{}
\item[Description]\AsIs{This repository contains the R package for Epidemiological Foci Relating Infections by Distance (EpiFRIenDs), a software to detect and analyse foci (clusters, outbreaks or hotspots) of infections from a given disease.}
\item[License]\AsIs{What license is it under?}
\item[Encoding]\AsIs{UTF-8}
\item[RoxygenNote]\AsIs{7.2.1}
\item[NeedsCompilation]\AsIs{no}
\end{description}
\Rdcontents{\R{} topics documented:}
\inputencoding{utf8}
\HeaderA{catalogue}{This method runs the DBSCAN algorithm (if cluster\_id is NULL) and obtains the mean positivity rate (PR) of each cluster extended with the non-infected cases closer than the link\_d.}{catalogue}
%
\begin{Description}\relax
This method runs the DBSCAN algorithm (if cluster\_id is NULL) and obtains the mean positivity rate (PR) of each cluster extended with the non-infected cases closer than the link\_d.
\end{Description}
%
\begin{Usage}
\begin{verbatim}
catalogue(
  positions,
  test_result,
  link_d,
  cluster_id = NULL,
  min_neighbours = 2,
  max_p = 1,
  min_pos = 2,
  min_total = 2,
  min_pr = 0
)
\end{verbatim}
\end{Usage}
%
\begin{Arguments}
\begin{ldescription}
\item[\code{positions}] data.frame with the positions of parameters we want to query with shape (n,2) where n is the number of positions.

\item[\code{test\_result}] data.frame with the test results (0 or 1).

\item[\code{link\_d}] The linking distance to connect cases. Should be in the same scale as the positions.

\item[\code{cluster\_id}] Numeric vector with the cluster IDs of each position, with 0 for those without a cluster. Give NULL if cluster\_id must be calculated.

\item[\code{min\_neighbours}] Minium number of neighbours in the radius < link\_d needed to link cases as friends.

\item[\code{max\_p}] Maximum value of the p-value to consider the cluster detection.

\item[\code{min\_pos}] Threshold of minimum number of positive cases in clusters applied.

\item[\code{min\_total}] Threshold of minimum number of cases in clusters applied.

\item[\code{min\_pr}] Threshold of minimum positivity rate in clusters applied.
\end{ldescription}
\end{Arguments}
%
\begin{Details}\relax
The epifriends package uses the RANN package which can gives the exact nearest neighbours using the friends of friends algorithm. For more information on the RANN library please visit https://cran.r-project.org/web/packages/RANN/RANN.pdf
\#'
\end{Details}
%
\begin{Value}
List with the next objects:
cluster\_id: numeric vector
Vector of the cluster IDs of each position, with 0 for those
without a cluster.
mean\_pr\_cluster: numeric vector
Mean PR corresponding to cluster\_id.
pval\_cluster: numeric vector
P-value corresponding to cluster\_id.
epifriends\_catalogue: List
Catalogue of the epifriends clusters and their main characteristics.
\end{Value}
%
\begin{Author}\relax
Mikel Majewski Etxeberria based on earlier python code by Arnau Pujol.
\end{Author}
%
\begin{Examples}
\begin{ExampleCode}
# Required packages
if(!require("RANN")) install.packages("RANN")
library("RANN")

# Creation of x vector of longitude coordinates, y vector of latitude coordinates and finaly merge them on a position data frame.
x <- c(1,2,3,4,7.5,8,8.5,9,10,13,13.1,13.2,13.3,14,15,30)
y <- c(1,2,3,4,7.5,8,8.5,9,10,13,13.1,13.2,13.3,14,15,30)
pos <- data.frame(x,y)

# Creation of test data frame with 0 for negative cases and 1 for positive clases for each position.
test <- data.frame(c(0,1,1,0,1,0,1,1,0,0,0,0,1,0,1,1))

# Creation of catalogue for this positions, linking distance 2 and default values.
cat <- catalogue(pos, test, 2)
\end{ExampleCode}
\end{Examples}
\inputencoding{utf8}
\HeaderA{dbscan}{This method finds the DBSCAN clusters from a set of positions and returns their cluster IDs.}{dbscan}
%
\begin{Description}\relax
This method finds the DBSCAN clusters from a set of positions and returns their cluster IDs.
\end{Description}
%
\begin{Usage}
\begin{verbatim}
dbscan(positions, link_d, min_neighbours = 2)
\end{verbatim}
\end{Usage}
%
\begin{Arguments}
\begin{ldescription}
\item[\code{positions}] data.frame with the positions of parameters we want to query with shape (n,2) where n is the number of positions.

\item[\code{link\_d:}] The linking distance to connect cases. Should be in the same scale as the positions.

\item[\code{min\_neighbours:}] Minium number of neighbours in the radius < link\_d needed to link cases as friends.
\end{ldescription}
\end{Arguments}
%
\begin{Details}\relax
The epifriends package uses the RANN package which gives us the exact nearest neighbours using the friends of friends algorithm. For more information on the RANN library please visit https://cran.r-project.org/web/packages/RANN/RANN.pdf
\end{Details}
%
\begin{Value}
cluster\_id: Vector of the cluster IDs of each position, with 0 for those without a cluster. Returns empty numeric vector if positions vector is empty.
\end{Value}
%
\begin{Author}\relax
Mikel Majewski Etxeberria based on earlier python code by Arnau Pujol.
\end{Author}
%
\begin{Examples}
\begin{ExampleCode}
# Required packages
if(!require("RANN")) install.packages("RANN")
library("RANN")

# Creation of x vector of longitude coordinates, y vector of latitude coordinates and finaly merge them on a position data frame.
x <- c(1,2,3,4,7.5,8,8.5,9,10,13,13.1,13.2,13.3,14,15,30)
y <- c(1,2,3,4,7.5,8,8.5,9,10,13,13.1,13.2,13.3,14,15,30)
pos <- data.frame(x,y)

# Computation of clusters of hotspots for positions with dbscan algorithm using linking distance 2 and minimum 3 neighbours.
db <- dbscan(pos, 2 ,3)
\end{ExampleCode}
\end{Examples}
\inputencoding{utf8}
\HeaderA{temporal\_catalogue}{This method generates a list of EpiFRIenDs catalogues representing different time frames by including only cases within a time window that moves within each time step.}{temporal.Rul.catalogue}
%
\begin{Description}\relax
This method generates a list of EpiFRIenDs catalogues representing different time frames by including only cases within a time window that moves within each time step.
\end{Description}
%
\begin{Usage}
\begin{verbatim}
temporal_catalogue(
  positions,
  test_result,
  dates,
  link_d,
  min_neighbours = 2,
  time_width,
  min_date = NULL,
  max_date = NULL,
  time_steps = 1,
  max_p = 1,
  min_pos = 2,
  min_total = 2,
  min_pr = 0,
  add_temporal_id = TRUE,
  linking_time,
  linking_dist,
  get_timelife = TRUE
)
\end{verbatim}
\end{Usage}
%
\begin{Arguments}
\begin{ldescription}
\item[\code{positions}] data.frame with the positions of parameters we want to query with shape (n,2) where n is the number of positions.

\item[\code{test\_result}] data.frame with the test results (0 or 1).

\item[\code{dates}] Vector of the date times of the corresponding data in format 'y-m-d h:m:s'.

\item[\code{link\_d}] the linking distance of the DBSCAN algorithm. Should be in the same scale as the positions.

\item[\code{min\_neighbours}] Minium number of neighbours in the radius < link\_d needed to link cases as friends.

\item[\code{time\_width}] Number of days of the time window used to select cases in each time step.

\item[\code{min\_date}] Initial date used in the first time step and time window selection. In format 'y-m-d h:m:s'.

\item[\code{max\_date}] Final date to analyse, defining the last time window as the one fully overlapping the data. In format 'y-m-d h:m:s'.

\item[\code{time\_steps}] Number of days that the time window is shifted in each time step.

\item[\code{max\_p}] Maximum value of the p-value to consider the cluster detection.

\item[\code{min\_pos}] Threshold of minimum number of positive cases in clusters applied.

\item[\code{min\_total}] Threshold of minimum number of cases in clusters applied.

\item[\code{min\_pr}] Threshold of minimum positivity rate in clusters applied.

\item[\code{add\_temporal\_id}] Boolean that indicates if we want to add a temporal id to the hotspots indicating which hotspots are near in time and space. Only posible if there 2 or more temporal windows.

\item[\code{linking\_time}] Maximum number of timesteps of distance to link hotspots with the same temporal ID. Only necesary if add\_temporal\_id is TRUE.

\item[\code{linking\_dist}] Linking distance used to link the clusters from the different time frames. Only necesary if add\_temporal\_id is TRUE.

\item[\code{get\_timelife}] It specifies if the time periods and timelife of clusters are obtained. Only necesary if add\_temporal\_id is TRUE.
\end{ldescription}
\end{Arguments}
%
\begin{Details}\relax
The epifriends package uses the RANN package which can gives the exact nearest neighbours using the friends of friends algorithm. For more information on the RANN library please visit https://cran.r-project.org/web/packages/RANN/RANN.pdf
\#'
\end{Details}
%
\begin{Value}
List with the next objects:
temporal\_catalogues: list of catalogues
List of EpiFRIenDs catalogues, where each element contains the catalogue in each
time step.
mean\_date: vector of chorn date times.
List of dates corresponding to the median time in each time window
\end{Value}
%
\begin{Author}\relax
Mikel Majewski Etxeberria based on earlier python code by Arnau Pujol.
\end{Author}
%
\begin{Examples}
\begin{ExampleCode}
# Required packages
if(!require("RANN")) install.packages("RANN")
if(!require("chron")) install.packages("chron")
library("RANN")
library("chron")

# Creation of x vector of longitude coordinates, y vector of latitude coordinates and finaly merge them on a position data frame.
x <- c(1,2,3,4,7.5,8,8.5,9,10,13,13.1,13.2,13.3,14,15,30)
y <- c(1,2,3,4,7.5,8,8.5,9,10,13,13.1,13.2,13.3,14,15,30)
pos <- data.frame(x,y)

# Creation of test data frame with 0 for negative cases and 1 for positive clases for each position.
test <- data.frame(c(0,1,1,0,1,0,1,1,0,0,0,0,1,0,1,1))

#Creation of chron dates vector in format 'y-m-d h:m:s'.
dates <- c("2020-11-03 05:33:07","2021-05-19 10:29:59","2021-02-09 14:53:20","2021-11-21 02:35:38","2020-11-19 05:57:24",
"2021-06-09 07:50:30","2021-09-18 05:53:26","2020-03-19 17:16:56","2021-06-08 12:40:46","2020-06-26 05:01:31",
"2020-10-15 04:40:27","2021-05-28 15:23:23","2020-05-01 02:56:54","2020-08-19 22:45:35","2021-10-23 18:56:35",
"2020-10-19 00:01:25")

# Creation of temporal catalogue for this data.
tcat <- tcat <- temporal_catalogue(pos, test, dates ,link_d = 2, time_width = 305, time_steps = 305, linking_time = 3, linking_dist = 2)
\end{ExampleCode}
\end{Examples}
\printindex{}
\end{document}
